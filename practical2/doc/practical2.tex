%%%%%%%%%%%%%%%%%%%%%%%%%%%%%%%%%%%%%%%%%
% Journal Article
% Distributed Parallel System
% Practical 2: MPI setup and sample code
%
% Gahan M. Saraiya
% 18MCEC10
%
% References
% ==========
% http://mpitutorial.com/tutorials/running-an-mpi-cluster-within-a-lan/
% https://www.codingame.com/playgrounds/349/introduction-to-mpi/measuring-time
% https://www.mcs.anl.gov/research/projects/mpi/tutorial/gropp/node139.html#Node139
% http://mpitutorial.com/tutorials/
% http://mpitutorial.com/tutorials/mpi-scatter-gather-and-allgather/
% 
%%%%%%%%%%%%%%%%%%%%%%%%%%%%%%%%%%%%%%%%%
%----------------------------------------------------------------------------------------
%       PACKAGES AND OTHER DOCUMENT CONFIGURATIONS
%----------------------------------------------------------------------------------------
\documentclass[paper=letter, fontsize=12pt]{article}
\usepackage[english]{babel} % English language/hyphenation
\usepackage{amsmath,amsfonts,amsthm} % Math packages
\usepackage[utf8]{inputenc}
\usepackage{xcolor}
\usepackage{float}
\usepackage{lipsum} % Package to generate dummy text throughout this template
\usepackage{blindtext}
\usepackage{graphicx} 
\usepackage{caption}
\usepackage{subcaption}
\usepackage[sc]{mathpazo} % Use the Palatino font
\usepackage[T1]{fontenc} % Use 8-bit encoding that has 256 glyphs
\usepackage{bbding}  % to use custom itemize font
\linespread{1.05} % Line spacing - Palatino needs more space between lines
\usepackage{microtype} % Slightly tweak font spacing for aesthetics
\usepackage[hmarginratio=1:1,top=32mm,columnsep=20pt]{geometry} % Document margins
\usepackage{multicol} % Used for the two-column layout of the document
%\usepackage[hang, small,labelfont=bf,up,textfont=it,up]{caption} % Custom captions under/above floats in tables or figures
\usepackage{booktabs} % Horizontal rules in tables
\usepackage{float} % Required for tables and figures in the multi-column environment - they need to be placed in specific locations with the [H] (e.g. \begin{table}[H])
\usepackage{hyperref} % For hyperlinks in the PDF
\usepackage{lettrine} % The lettrine is the first enlarged letter at the beginning of the text
\usepackage{paralist} % Used for the compactitem environment which makes bullet points with less space between them
\usepackage{abstract} % Allows abstract customization
\renewcommand{\abstractnamefont}{\normalfont\bfseries} % Set the "Abstract" text to bold
\renewcommand{\abstracttextfont}{\normalfont\small\itshape} % Set the abstract itself to small italic text
\usepackage{titlesec} % Allows customization of titles

%Importing csv as table
\usepackage{csvsimple}

\usepackage{makecell}
\usepackage{longtable}
\renewcommand\thesection{\Roman{section}} % Roman numerals for the sections
\renewcommand\thesubsection{\Roman{subsection}} % Roman numerals for subsections
%----------------------------------------------------------------------------------------
%       DATE FORMAT
%----------------------------------------------------------------------------------------
\usepackage{datetime}
\newdateformat{monthyeardate}{\monthname[\THEMONTH], \THEYEAR}
%----------------------------------------------------------------------------------------

\titleformat{\section}[block]{\large\scshape\centering}{\thesection.}{1em}{} % Change the look of the section titles
\titleformat{\subsection}[block]{\large}{\thesubsection.}{1em}{} % Change the look of the section titles
\newcommand{\horrule}[1]{\rule{\linewidth}{#1}} % Create horizontal rule command with 1 argument of height
\usepackage{fancyhdr} % Headers and footers
\pagestyle{fancy} % All pages have headers and footers
\fancyhead{} % Blank out the default header
\fancyfoot{} % Blank out the default footer


%----------------------------------------------------------------------------------------
%       TITLE SECTION
%----------------------------------------------------------------------------------------
\title{\vspace{-15mm}\fontsize{24pt}{10pt}\selectfont\textbf{Practical 2: MPI setup and sample code}} % Article title
\author{
\large
{\textsc{Gahan Saraiya (18MCEC10)}}\\[2mm]
%\thanks{A thank you or further information}\\ % Your name
\normalsize \href{mailto:18mcec10@nirmauni.ac.in}{18mcec10@nirmauni.ac.in}\\[2mm] % Your email address
}
\date{}
\hypersetup{
	colorlinks=true,
	linkcolor=blue,
	filecolor=magenta,      
	urlcolor=cyan,
	pdfauthor={Gahan Saraiya},
	pdfcreator={Gahan Saraiya},
	pdfproducer={Gahan Saraiya},
}
%----------------------------------------------------------------------------------------

%----------------------------------------------------------------------------------------
%       SET HEADER AND FOOTER
%----------------------------------------------------------------------------------------
\newcommand\theauthor{Gahan Saraiya}
\newcommand\thesubject{Distributed Parallel System}
\renewcommand{\footrulewidth}{0.4pt}% default is 0pt
\fancyhead[C]{Institute of Technology, Nirma University $\bullet$ \monthyeardate\today} % Custom header text
\fancyfoot[LE,LO]{\thesubject}
\fancyfoot[RO,LE]{Page \thepage} % Custom footer text
%----------------------------------------------------------------------------------------

\usepackage[utf8]{inputenc}
\usepackage[english]{babel}
\usepackage[utf8]{inputenc}
\usepackage{fourier} 
\usepackage{array}
\usepackage{makecell}

\renewcommand\theadalign{bc}
\renewcommand\theadfont{\bfseries}
\renewcommand\theadgape{\Gape[4pt]}
\renewcommand\cellgape{\Gape[4pt]}
\newcommand*\tick{\item[\Checkmark]}
\newcommand*\arrow{\item[$\Rightarrow$]}
\newcommand*\fail{\item[\XSolidBrush]}
\usepackage{minted} % for highlighting code sytax
\definecolor{LightGray}{gray}{0.9}

\setminted[text]{
	frame=lines, 
	breaklines,
	baselinestretch=1.2,
	bgcolor=LightGray,
%	fontsize=\small
}
\setminted[bash]{
%	frame=lines, 
	breaklines,
	baselinestretch=1.2,
	bgcolor=LightGray,
%	fontsize=\small
}

\setminted[python]{
	frame=lines, 
	breaklines, 
	linenos,
	baselinestretch=1.2,
%	bgcolor=LightGray,
%	fontsize=\small
}
\setminted[c]{
	frame=lines, 
	breaklines, 
	linenos,
	baselinestretch=1.2,
%	bgcolor=LightGray,
%	fontsize=\small
}

\begin{document}
\maketitle % Insert title
\thispagestyle{fancy} % All pages have headers and footers

\section{AIM}
\begin{itemize}
	\item Setup MPI
	\begin{itemize}
		\item Write steps for MPI setup
		\item List problem observed during setup and execution
		\item Specify cause and solution of each problem
	\end{itemize}
    \item print Computer machine name and rank id for each process.
    \item MPI time function to measure time taken by given fragment.
    \item Communication function (send/receive or scatter or broadcast etc.)
    \item Find command to set fix process per node.
\end{itemize}

\section{Prerequisite}
Install MPICH2 in each of the machine

\section{MPI Setup}
\subsection{Configure hosts file}
You are gonna need to communicate between the computers and you don’t want to type in the IP addresses every so often. Instead, you can give a name to the various nodes in the network that you wish to communicate with. hosts file is used by your device operating system to map hostnames to IP addresses.

\begin{minted}{bash}
$ sudo nano /etc/hosts

10.1.3.4       master
10.1.3.5       node
\end{minted}

\subsection{Create a new user}
Though you can operate your cluster with your existing user account, I’d recommend you to create a new one to keep our configurations simple. 
Let us create a new user $ mpi $.

\textit{Create new user accounts with the same username in all the machines to keep things simple.}

\begin{minted}{bash}
$ sudo adduser mpi
\end{minted}

Follow prompts and you will be good. Please don’t use $ useradd $ command to create a new user as that doesn’t create a separate home for new users.

\subsection{Setting up SSH}
machines are gonna be talking over the network via SSH and share data via NFS.

\begin{minted}{bash}
$ sudo apt­-get install openssh-server
\end{minted}

And right after that, login with your newly created account
\begin{minted}{bash}
$ su - mpi
\end{minted}
 
Since the ssh server is already installed, you must be able to login to other machines by ssh username@hostname, at which you will be prompted to enter the password of the username. To enable more easier login, we generate keys and copy them to other machines’ list of authorized\_keys.

\begin{minted}{bash}
$ ssh-keygen -t rsa
\end{minted}
Now, add the generated key to each of the other computers. In our case, the client machine.

\begin{minted}{bash}
$ ssh-copy-id client #ip-address may also be used
\end{minted}

Do the above step for each of the client machines and your own user (localhost).

This will setup openssh-server for you to securely communicate with the client machines. ssh all machines once, so they get added to your list of known\_hosts. This is a very simple but essential step failing which passwordless ssh will be a trouble.

Now, to enable passwordless ssh,

\begin{minted}{bash}
$ eval `ssh-agent`
$ ssh-add ~/.ssh/id_dsa
\end{minted}

Now, assuming you’ve properly added your keys to other machines, you must be able to login to other machines without any password prompt.


\subsection{Setting up NFS}
You share a directory via NFS in \textbf{master} which the \textbf{mpi} mounts to exchange data.

\subsubsection{NFS-Server}
Install the required packages by
\begin{minted}{bash}
$ sudo apt-get install nfs-kernel-server
\end{minted}

Now export directory to share among nodes as below:
\begin{minted}{bash}
$ sudo nano /etc/exports
/home/mpiuser/cloud *(rw,sync,no_root_squash,no_subtree_check)
\end{minted}

\begin{itemize}
    \item Here, instead of $ * $ you can specifically give out the IP address to which you want to share this folder to. But, this will just make our job easier.
    
    \item \textbf{rw}: This is to enable both read and write option. ro is for read-only.
    \item \textbf{sync}: This applies changes to the shared directory only after changes are committed.
    \item \textbf{no\_subtree\_check}: This option prevents the subtree checking. When a shared directory is the subdirectory of a larger filesystem, nfs performs scans of every directory above it, in order to verify its permissions and details. Disabling the subtree check may increase the reliability of NFS, but reduce security.
    \item \textbf{no\_root\_squash}: This allows root account to connect to the folder.
\end{itemize}

After you have made the entry, run the following.

\begin{minted}{bash}
$ sudo exportfs -a
\end{minted}
Run the above command, every time you make a change to $ /etc/exports $.


If required, restart the nfs server
\begin{minted}{bash}
$ sudo service nfs-kernel-server restart
\end{minted}

\subsubsection{NFS-Client}
Install the required packages
\begin{minted}{bash}
$ sudo apt-get install nfs-common
\end{minted}

mount the shared directory like
\begin{minted}{bash}
$ sudo mount -t nfs master:/home/mpi ~/
\end{minted}

To make the mount permanent so you don’t have to manually mount the shared directory everytime you do a system reboot, you can create an entry in your file systems table - i.e., $ /etc/fstab $ file like this:
\begin{minted}{bash}
$ sudo nano /etc/fstab
#MPI CLUSTER SETUP
master:/home/mpiuser/cloud /home/mpiuser/cloud nfs
\end{minted}

\section{Troubleshooting Setup}
\begin{itemize}
    \item Make sure all the machines you are trying to run the executable on, has the same version of MPI. Recommended is MPICH2.
    \item The hosts file of master should contain the local network IP address entries of master and all of the slave nodes. For each of the slave, you need to have the IP address entry of master and the corresponding slave node.
    \item Whenever you try to run a process parallely using MPI, you can either run the process locally or run it as a combination of local and remote nodes. You cannot invoke a process only on other nodes.
    
    To make this more clear, from master node, this script can be invoked.
    \begin{minted}{bash}
$ mpirun -np 10 --hosts master ./main
# To run the program only on the same master node
    \end{minted}
    So can this be. The following will also run perfectly.
    \begin{minted}{bash}
$ mpirun -np 10 --hosts master,node ./main
# To run the program on master and slave nodes.
    \end{minted}
    But, the following is not correct and will result in an error if invoked from master.
    \begin{minted}{bash}
$ mpirun -np 10 --hosts node ./cpi
# Trying to run the program only on remote slave
    \end{minted}
    \item \textbf{Architecture Issue}: Try using same hardware architecture for master and slaves (otherwise you need to learn how to run mpi file on cross platform) and will generate either error with using library with either ELFCLASS64 or ELFCLASS32
    \item Avoid mount entry in $ /etc/fstab $
    \\ The reason is if your server is not going to be live 24 hours the system tries to mount that partition and you might face login loop for the user.
    \item DO NOT TURN OFF master node before un mounting partition which is mounted in home directory of existing system 
\end{itemize}


\subsection{Sequential Code}
\inputminted{c}{../src/main.c}

\subsubsection{Compiling and Executing File!!!}
\inputminted{bash}{../src/run.sh}

\subsubsection{Output}
\inputminted{text}{../src/output.txt}

\section{Fix Process per node}
According to \href{https://www.open-mpi.org/doc/v3.1/man1/mpirun.1.php}{documentation of mpirun} here are the list of flags helpful to set fix number of processes per node.

\begin{itemize}
    \item \textbf{-npersocket, --npersocket <\#persocket>}
    On each node, launch this many processes times the number of processor sockets on the node. The -npersocket option also turns on the -bind-to-socket option. (deprecated in favor of --map-by ppr:n:socket) 
    \item \textbf{-npernode, --npernode <\#pernode>}
    On each node, launch this many processes. (deprecated in favor of --map-by ppr:n:node) 
    \item \textbf{-pernode, --pernode}
    On each node, launch one process -- equivalent to -npernode 1. (deprecated in favor of --map-by ppr:1:node)
\end{itemize}
The number of processes launched can be specified as a multiple of the number of nodes or processor sockets available. 
 
For example,
\begin{minted}{bash}
mpirun -H aa,bb -npersocket 2 ./a.out
\end{minted}
launches processes 0-3 on node aa and process 4-7 on node bb, where aa and bb are both dual-socket nodes. The -npersocket option also turns on the -bind-to-socket option, which is discussed in a later section.
 
\begin{minted}{bash}
mpirun -H aa,bb -npernode 2 ./a.out
\end{minted}
launches processes 0-1 on node aa and processes 2-3 on node bb. 

\begin{minted}{bash}
mpirun -H aa,bb -npernode 1 ./a.out
\end{minted}
launches one process per host node. 

\begin{minted}{bash}
mpirun -H aa,bb -pernode ./a.out
\end{minted}
is the same as -npernode 1. 

\end{document}

